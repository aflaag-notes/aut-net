\documentclass[a4paper, 12pt]{report}

\usepackage[dvipsnames]{xcolor}

%%%%%%%%%%%%%%%%
% Set Variables %
%%%%%%%%%%%%%%%%

\def\useItalian{0}  % 1 = Italian, 0 = English

\def\courseName{Course Name}

\def\coursePrerequisites{Small list of prerequisites}

\def\book{"My book",\\Author 1, ...}

\def\authorName{Simone Bianco}
\def\email{bianco.simone@outlook.it}
\def\github{https://github.com/Exyss/university-notes}
\def\linkedin{https://www.linkedin.com/in/simone-bianco}

% \def\authorName{Alessio Bandiera}
% \def\email{alessio.bandiera02@gmail.com}
% \def\github{https://github.com/aflaag-notes}
% \def\linkedin{https://www.linkedin.com/in/alessio-bandiera-a53767223}

% Do not change

%%%%%%%%%%%%
% Packages %
%%%%%%%%%%%%

\usepackage{../../packages/Nyx/nyx-packages}
\usepackage{../../packages/Nyx/nyx-styles}
\usepackage{../../packages/Nyx/nyx-frames}
\usepackage{../../packages/Nyx/nyx-macros}
\usepackage{../../packages/Nyx/nyx-title}
\usepackage{../../packages/Nyx/nyx-intro}

%%%%%%%%%%%%%%
% Title-page %
%%%%%%%%%%%%%%

\logo{../../packages/Nyx/logo.png}

\if\useItalian1
    \institute{\curlyquotes{\hspace{0.25mm}Sapienza} Università di Roma}
    \faculty{Ingegneria dell'Informazione,\\Informatica e Statistica}
    \department{Dipartimento di Informatica}
    \ifdefined\book
        \subtitle{Appunti integrati con il libro \book}
    \fi
    \author{\textit{Autore}\\\authorName}
\else
    \institute{\curlyquotes{\hspace{0.25mm}Sapienza} University of Rome}
    \faculty{Faculty of Information Engineering,\\Informatics and Statistics}
    \department{Department of Computer Science}
    \ifdefined\book
        \subtitle{Lecture notes integrated with the book \book}
    \fi
    \author{\textit{Author}\\\authorName}
\fi

\title{\courseName}
\date{\today}

% \supervisor{Linus \textsc{Torvalds}}
% \context{Well, I was bored\ldots}

\addbibresource{./references.bib}

%%%%%%%%%%%%
% Document %
%%%%%%%%%%%%

\begin{document}
\maketitle

% The following style changes are valid only inside this scope 
{
	\hypersetup{allcolors=black}
	\fancypagestyle{plain}{%
		\fancyhead{}        % clear all header fields
		\fancyfoot{}        % clear all header fields
		\fancyfoot[C]{\thepage}
		\renewcommand{\headrulewidth}{0pt}
		\renewcommand{\footrulewidth}{0pt}}

	\romantableofcontents
}

\introduction

%%%%%%%%%%%%%%%%%%%%%

\chapter{Q-routing literature review}

TODO \todo{intro}

In RL based design, the following aspects are addressed:

\begin{itemize}
	\item identification of the most appropriate states and actions of the agent
	\item definition of the reward function depending on the metrics to optimize
	\item identification of environment model when available
\end{itemize}

Given a target field of application, different design models may be elaborated, which differ in how they address each of the previous aspects.

In the last 25 years many RL-based routing protocol have been proposed, but most of them share the same high-level structure.

In literature, nodes are confused with agents, and in almost all protocols the reward is (at least partially) calculated by a node upon selecting an entire route to use for all packtes to transmit, or just a next hop to transmit the current data packet. Thus, a \tit{node} should be considered to consist of an agent and optional \tit{modules}:

\begin{itemize}
	\item \tbf{local reward} module: it calculates reward based on local view, which reflects the cost of communication as seen by the packet sender
	\item \tbf{remote reward} module: it receives feedback sent by the next hop or by the destinatino node --- if local and remote modules are both employed they are combined to form the reward return to the agent
	\item \tbf{link-state information maintenance} module: it collects useful link state information through periodi or on-demand \tit{Hello packets}
\end{itemize}

Therefore, the neighboring nodes of a node define the environment of the agent representing a node.

TODO \todo{parlare dei tipi di reti?}

\section{Q-routing}

\textcite{qlearning} were the first to propose a hop-by-bop routing algorithm based on Q-learning, called Q-routing, and the most of exists RL-based routing protocols today are just extensions of this algorithm. The following is the algorithm that defines Q-routing in detail.

% \begin{framedalgo}{Q-routing}
\begin{algorithmic}[1]
	\Function{Qrouting}{ }
	\State Initialize $Q_i$ matrix randomly
	\While{termination condition holds}
	\If{packet $P$ is ready to be sent to $d$}
	\State Determine node $j^* \gets \argmin_{j \in \mathcal N (i)}{Q_i(d, j)}$
	\State Send packet to node $j^*$
	\State Collect estimate $\theta_{j^*}(d)$ from node $j^*$
	\State Update $Q_i(d, j^*) \gets (1 - \alpha) \cdot Q_i(d, j^*) + \alpha \cdot \sbk{W_i^q(P) + T_{ij^*} + \theta_{j^*}(d)}$
	\EndIf
	\EndWhile
	\EndFunction
\end{algorithmic}
% \end{framedalgo}

We will briefly explain the algorithm. First, let's present the notation:

\begin{itemize}
	\item $i$ is the node that is currently running the algorithm
	\item $P$ is a packet that node $i$ needs to forward to destination $d$
	\item $Q_i(d, j)$ is the \tit{delivery delay} that $i$ estimates it takes, for node $j$, to deliver the packet $P$ at destination $i$
	\item $\mathcal N(j)$ is the set of $j$'s neighbors
	\item $\theta_j(d)$ is $j$'s estimate for the time remaining in the trip to destination $d$ of packet $P$
	\item $W_i^q(P)$ is the time spent by packet $P$ in node $i$'s queue
	\item $T_{i j}$ is the transmission time between nodes $i$ and $j$
\end{itemize}

Each entry of the table $Q_i$ is called \tbf{Q-value}, and when node $i$ wants to send a packed, it selects the node $j^*$ that minimizes the Q-value. Upon sendin packet $P$ to node $j^*$, node $i$ receives back from $j^*$ the value $$\theta_{j^*}(d) = \min_{k \in \mathcal N(j^*)}{Q_{j^*}(d, k)}$$ Then, node $i$ has to update its estimate of $Q_i(d, j^*)$ based on $\theta_{j^*}(d)$, which can be performed utilizing the formula $$Q(s_t, a_t) = (1 - \alpha) \cdot Q(s_t, a_t) + \alpha \cdot \sbk{R_{t + 1} + \gamma \cdot \max_{a \in A}{Q(s_{t + 1}, a)}}$$ by setting

\begin{itemize}
	\item $R_{t + 1} = W_i^q(P) + T_{ij^*}$ since it represents the \tit{link cost}
	\item $\gamma = 1$
	\item $\max_{a \in A}{Q(s_{t + 1}, a)} = \min_{k \in \mathcal N(j^*)}{Q_{j^*}(d, k)}$ since the \curlyquotes{action to take} corresponds to choosing an neighbor in this context, however we seek to \tit{minimize} the Q-value since the delay is clearly a decreasing metric
\end{itemize}

Despite the wide adoption of this protocol over the years, Q-routing is still far from perfect and it has its flaws. Some of the problems are inherent problems of Q-learning in general, such as \tit{slow convergence} rate and high \tit{parameter setting sensitivity}, but there are problem that arise from the protocol itself. For instance, to avoid frequen oscillations of the Q-values --- in case of sudden variations of traffic in the network --- and limit the overhead of the protocol, \textcite{nowe} proposed an extension in which $j^*$ sends an average $\overline \theta_{j^*}(d)$ to $i$ only after a certain number of exchanged packets. Another known problem is called \tbf{Q-value freshness}: the estimate $\theta_j(d)$ is evaluated upon packet transmission on a route, therefore if a route is not selected during a long period of time, the agent does not have an accurate estimate of the current condition of such route.

\chapter{Classification criteria}

The authors underline that, to their knowledge, their work is the first in the literature that proposes classification criteria to help understanding and comparing all the available RL-based routing protocols. These criteria are divided into 3 groups:

\begin{itemize}
	\item \tbf{context of use}: these are criteria that describe the targeted applications and their characteristics and requirements
	\item \tbf{design characteristics}: criteria in this group highlight how authors designed their protocols to make them efficient and different from the others
	\item \tbf{performance}: in this last category, criteria provide a qualitative evaluation of the overhead of protocols and the metrics used by the authors
\end{itemize}

We will cover each criteria in the

\section{Context of use}

\subsection{Network class and Assumptions}

TODO \todo{non ho capito}

\subsection{Routing Optimization Context}

From users' perspective, routing protocols should always be able to determine and select the optimal paths to convey data from sources to destinations. There are different ways to achieve such goal, that depend on

\begin{itemize}
	\item roles assigned to data sources
	\item roles assigned to relaying nodes
	\item initial assumptions about routing
\end{itemize}

The authors outlined 6 different \tit{routing optimization contexts}, which we will briefly explain one by one.

\begin{enumerate}
	\item \tbf{Data-packet driven optimazion}: in this context the transmission of packets happens hop-by-bop from source $s$ to destination $d$, and upon receival $d$ sends back a feedback. After a given amount of forwarded packets, the routing process converges to the selectoin of optimal paths.
	\item \tbf{Route requrest driven optimization}: a source $s$ that has data to send to $d$, first sends a Route Request (RR) packet. The latter is then disseminated in the network, and each node that receives the RR packet can decide to partecipate or not --- if it agrees to partecipate, it selects the next node to forward the RR packet to, and this process continues until $d$ is reached. Once a path is found, all packets from $s$ to $d$ are routed through this path. Then, at the end of each transmission a feedback is sent back to the sender regarding the performance of current nodes. Most protocols in this category are extensions to the \tbf{AODV protocol}
\end{enumerate}

\addcontentsline{toc}{chapter}{Bibliography}  % Add Bibliography to ToC
\printbibliography % UNCOMMENT FOR BIBLIOGRAPHY

\end{document}
